\documentclass[a4paper,11pt, onecolumn,twoside]{article}

\usepackage{alltt}
\usepackage{polyglossia}
\setdefaultlanguage[variant=australian]{english}

\usepackage{fontspec}
\usepackage{xunicode}

\setmainfont{Times New Roman}
\newfontfamily\ll[]{Linux Libertine O}
%\newfontfamily\udfont[Scale=1.3,Letters=SmallCaps]{Linux Libertine O}
\newfontfamily\udfont[Scale=1.3]{Linux Libertine O}
\newfontfamily\fsfont[Scale=MatchLowercase]{FreeSerif}

\usepackage[normalem]{ulem}
\usepackage{multirow}
\usepackage{multicol}
\usepackage[small,bf]{caption} % CHECK IF THIS IS OK
\usepackage[colorlinks=true,citecolor=black,linkcolor=black,urlcolor=blue]{hyperref}

\usepackage{subcaption}
\usepackage{booktabs}

\usepackage{tikz-dependency}

\newcommand{\gmk}[1]{{\rm {\ll \textsc{#1}}}}
\newcommand{\kazakh}[1]{{\em #1}}
\newcommand{\gloss}[1]{`#1'}
\newcommand{\sgloss}[1]{\hspace{0.5em}`#1'}
%\newcommand{\sgloss}[1]{\hfill{}`#1'}
\newcommand{\tgloss}[1]{{\em #1}}
%\newcommand{\tag}[1]{\texttt{‹#1›}}
\newcommand{\tag}[1]{{\ll \textsc{‹#1›}}}
%\newcommand{\udtag}[1]{\texttt{@#1}}
\newcommand{\udtag}[1]{{\ll \textsc{#1}}}
\newcommand{\udlabel}[1]{{\udfont #1}}
\newcommand{\tilda}{{\fsfont ∼}}



\begin{document}

\section{Morphological categories}

\section{Parts-of-speech}

The parts-of-speech in use are described in more detail here.\footnote{\url{http://universaldependencies.org/ru/pos/all.html}}

Open class words:

\begin{tabular}{ll}
   \gmk{adj} &    \\
   \gmk{adv} & \emph{Германские войска \underline{быстро} продвигались вперёд и к 1942 году вышли к Северному Кавказу.}  \\
   \gmk{intj} &  \\
   \gmk{noun} &  \\
   \gmk{propn} &  \\
   \gmk{verb} &  \\
\end{tabular}

Closed class words:

\begin{itemize}
   \item \gmk{adp}
   \item \gmk{conj}
   \item \gmk{det}
   \item \gmk{num}
   \item \gmk{part}
   \item \gmk{pron}
   \item \gmk{sconj}
\end{itemize}

Other:

\begin{itemize}
   \item \gmk{punct}
   \item \gmk{x}
\end{itemize}

\section{Phrasal categories}

\section{Dependency relations}

The dependency relations in the constructicon are based on Universal Dependencies.\footnote{\url{http://}}

\begin{figure}
\centering
        \begin{dependency}[theme = simple, font = \small]
           \begin{deptext}[column sep=0.2cm]
                         \_ \& с \& \_                \\
                         \gmk{pron} \& \gmk{adp} \& \_        \\
                         \gmk{PronType=Pers|Number=Plur} \& \_ \& \gmk{Case=Ins} \\
                \end{deptext}
                \deproot[edge unit distance=1.3ex]{1}{\udlabel{x}}
                \depedge{3}{2}{\udlabel{case}}
                \depedge{1}{3}{\udlabel{nmod}}
        \end{dependency}
\end{figure}

\section{Semantic roles}


\section{Naming constructions}

% when X/Y/NP/etc.

\section{Structure of constructicon}

% what are "Keywords": parts of the construction

% what are "Common words": words found very frequently with this construction

% 


\end{document}
