\documentclass[a4paper,11pt, onecolumn,twoside]{article}

\usepackage{alltt}
\usepackage{polyglossia}
\setdefaultlanguage[variant=australian]{english}

\usepackage{fontspec}
\usepackage{xunicode}
\usepackage{longtable}

\setmainfont{Times New Roman}
\newfontfamily\ll[]{Linux Libertine O}
%\newfontfamily\udfont[Scale=1.3,Letters=SmallCaps]{Linux Libertine O}
\newfontfamily\udfont[Scale=1.3]{Linux Libertine O}
\newfontfamily\fsfont[Scale=MatchLowercase]{FreeSerif}

\usepackage[normalem]{ulem}
\usepackage{multirow}
\usepackage{multicol}
\usepackage[small,bf]{caption} % CHECK IF THIS IS OK
\usepackage[colorlinks=true,citecolor=black,linkcolor=black,urlcolor=blue]{hyperref}

\usepackage{subcaption}
\usepackage{booktabs}

\usepackage{tikz-dependency}

\newcommand{\gmk}[1]{{\rm {\ll \textsc{#1}}}}
\newcommand{\kazakh}[1]{{\em #1}}
\newcommand{\gloss}[1]{`#1'}
\newcommand{\sgloss}[1]{\hspace{0.5em}`#1'}
%\newcommand{\sgloss}[1]{\hfill{}`#1'}
\newcommand{\tgloss}[1]{{\em #1}}
%\newcommand{\tag}[1]{\texttt{‹#1›}}
\newcommand{\tag}[1]{{\ll \textsc{‹#1›}}}
%\newcommand{\udtag}[1]{\texttt{@#1}}
\newcommand{\udtag}[1]{{\ll \textsc{#1}}}
\newcommand{\udlabel}[1]{{\udfont #1}}
\newcommand{\tilda}{{\fsfont ∼}}



\begin{document}

\section{Morphological categories}

\section{Parts-of-speech}

The parts-of-speech in use are described in more detail here.\footnote{\url{http://universaldependencies.org/ru/pos/all.html}}

Open class words:

\begin{tabular}{ll}
   \gmk{adj} &    \\
   \multirow{2}{*}{\gmk{adv}} & «\emph{Германские войска \underline{быстро} продвигались вперёд} \\
                              & ~~\emph{и к 1942 году вышли к Северному Кавказу.}» \\
   \gmk{intj} &  \\
   \gmk{noun} &  \\
   \gmk{propn} &  \\
   \gmk{verb} &  \\
\end{tabular}

Closed class words:

\begin{itemize}
   \item \gmk{adp}
   \item \gmk{conj}
   \item \gmk{det}
   \item \gmk{num}
   \item \gmk{part}
   \item \gmk{pron}
   \item \gmk{sconj}
\end{itemize}

Other:

\begin{itemize}
   \item \gmk{punct}
   \item \gmk{x}
\end{itemize}

\section{Phrasal categories}

\begin{tabular}{lll}

\multirow{2}{*}{NP} & & \\
                    & & \\
\multirow{2}{*}{VP} & & \\
                    & & \\
\multirow{2}{*}{AP} & & \\
                    & & \\
\multirow{2}{*}{AdvP} & & \\
                    & & \\
\multirow{2}{*}{PP} & & \\
                    & & \\
\multirow{2}{*}{NumP} & & \\
                    & & \\
\multirow{2}{*}{XP} & & \\
                    & & \\
\multirow{2}{*}{Cl} & & \\
                    & & \\
\multirow{2}{*}{S} & & \\
                    & & \\
\end{tabular}

\section{Dependency relations}

The dependency relations in the constructicon are based on Universal Dependencies.\footnote{\url{http://}}

\begin{figure}
\centering
        \begin{dependency}[theme = simple, font = \small]
           \begin{deptext}[column sep=0.2cm]
                     ~ \&   \_ \& с \& \_                \\
                     POS:  \&  \gmk{PRON} \& \gmk{ADPOS} \& \_        \\
                     PronType:  \& \gmk{Pers} \& \_ \& \_ \\
                     Number: \& \gmk{Plur} \& \_ \& \_ \\
                     Case: \& \_ \& \_ \& \gmk{Ins} \\
                \end{deptext}
                \deproot[edge unit distance=1.3ex]{2}{\udlabel{x}}
                \depedge{4}{3}{\udlabel{case}}
                \depedge{2}{4}{\udlabel{nmod}}
        \end{dependency}
\end{figure}

\section{Semantic roles}
\label{sec:semroles}

Definitions of semantic roles have been taken from various sources.
% http://www-01.sil.org/linguistics/GlossaryOfLinguisticTerms/

\begin{longtable}{ p{.20\textwidth}  p{.80\textwidth} } 
\toprule
 \multirow{2}{*}{Actant} & ~ \\ 
        & ~ \\
\midrule
 \multirow{2}{*}{Action} & ~ \\ 
        & ~ \\
\midrule
 \multirow{2}{*}{Activity} & ~ \\ 
        & ~ \\
\midrule
 \multirow{3}{*}{Agent} & Refers to the initiator of an event. \\ 
        & [Вася]$_{Agent}$ ловит рыбу. \\
        & Рыба ловится [Васей]$_{Agent}$ \\
\midrule
 \multirow{2}{*}{Associated} & ~ \\ 
        & ~ \\
\midrule
 \multirow{2}{*}{BaseOfPredication} & ~ \\ 
        & ~ \\
\midrule
 \multirow{2}{*}{Beneficiary} & Refers to a referent which is advantaged or disadvantaged by an event. \\ 
        & Когда Саше было девять лет, мать купила [ему]$_{Beneficiary}$ кроссовки. \\
\midrule
 \multirow{2}{*}{Cause} & ~ \\ 
        & ~ \\
\midrule
 \multirow{2}{*}{Causer} & Refers to the referent which instigates an event rather than actually doing it. \\ 
        & ~ \\
\midrule
 \multirow{2}{*}{Circumstance} & ~ \\ 
        & ~ \\
\midrule
 \multirow{2}{*}{Class} & ~ \\ 
        & ~ \\
\midrule
 \multirow{2}{*}{Concerning} & ~ \\ 
        & ~ \\
\midrule
 \multirow{2}{*}{Condition} & ~ \\ 
        & ~ \\
\midrule
 \multirow{2}{*}{Context} & ~ \\ 
        & ~ \\
\midrule
 \multirow{2}{*}{Corresponding} & ~ \\ 
        & ~ \\
\midrule
 \multirow{2}{*}{Direction} & ~ \\ 
        & ~ \\
\midrule
 \multirow{2}{*}{Distance} & ~ \\ 
        & ~ \\
\midrule
 \multirow{2}{*}{Entity} & ~ \\ 
        & ~ \\
\midrule
 \multirow{2}{*}{Evaluation} & ~ \\ 
        & ~ \\
\midrule
 \multirow{2}{*}{Event} & ~ \\ 
        & ~ \\
\midrule
 \multirow{2}{*}{Experiencer} & Refers to an entity that receives a sensory impression, or in some other way is the locus of some event or activity that involves neither volition nor a change of state. \\ 
        & ~ \\
\midrule
 \multirow{2}{*}{Function} & ~ \\ 
        & ~ \\
\midrule
 \multirow{2}{*}{Goal} & Refers to the place to which something moves, or the thing towards which an action is directed. \\ 
        & ~ \\
\midrule
 \multirow{2}{*}{Instrument} & Refers to an inanimate thing that an agent uses to implement an event. It is the stimulus or immediate physical cause of an event. \\ 
        & Он написал пиьсмо [карандашом]$_{Instrument}$. \\
\midrule
 \multirow{2}{*}{Landmark} & ~ \\ 
        & ~ \\
\midrule
 \multirow{2}{*}{Location} & ~ \\ 
        & ~ \\
\midrule
 \multirow{2}{*}{Locative} & Refers to the location or spatial orientation of a state or action. It does not imply any motion. \\ 
        & Иван нашел лягушку [в лесу]$_{Locative}$. \\
\midrule
 \multirow{2}{*}{Manner} & Notes how the action, experience, or process of an event is carried out. \\ 
        & ~ \\
\midrule
 \multirow{2}{*}{Material} & ~ \\ 
        & ~ \\
\midrule
 \multirow{2}{*}{Measure} & Notes the quantification of an event. \\ 
        & ~ \\
\midrule
 \multirow{2}{*}{Object} & ~ \\ 
        & ~ \\
\midrule
 \multirow{2}{*}{Part} & ~ \\ 
        & ~ \\
\midrule
 \multirow{2}{*}{Participant} & ~ \\ 
        & ~ \\
\midrule
 \multirow{2}{*}{Path} & ~ \\ 
        & ~ \\
\midrule
 \multirow{2}{*}{Patient} & Refers to the surface object of the verb in a sentence. \\ 
        & ~ \\
\midrule
 \multirow{2}{*}{Phenomenon} & ~ \\ 
        & ~ \\
\midrule
 \multirow{2}{*}{Property} & ~ \\ 
        & ~ \\
\midrule
 \multirow{2}{*}{Purpose} & ~ \\ 
        & ~ \\
\midrule
 \multirow{2}{*}{Quantitative} & ~ \\ 
        & ~ \\
\midrule
 \multirow{2}{*}{Recipient} & Refers to a referent that is conscious of being affected by the state or action identified by the verb. \\ 
        & ~ \\
\midrule
 \multirow{2}{*}{Requirement} & ~ \\ 
        & ~ \\
\midrule
 \multirow{2}{*}{Result} & ~ \\ 
        & ~ \\
\midrule
 \multirow{2}{*}{Situation} & ~ \\ 
        & ~ \\
\midrule
 \multirow{2}{*}{Source} & Refers to the (a) the place of origin, (b) the entity from which a physical sensation comes, (c) the original owner in a transfer. \\ 
        & ~ \\
\midrule
 \multirow{2}{*}{State} & ~ \\ 
        & ~ \\
\midrule
 \multirow{2}{*}{Theme} & ~ \\ 
        & ~ \\
\midrule
 \multirow{2}{*}{Trajector} & ~ \\ 
        & ~ \\
\midrule
 \multirow{2}{*}{Undergoer} & ~ \\ 
        & ~ \\
\bottomrule
\caption{List of semantic roles}
\label{table:semroles}
\end{longtable}

\section{Naming constructions}
\label{sec:name}

% when X/Y/NP/etc.

\section{The structure of the constructicon}

\begin{description}
  \item[Name] This is the name of the construction (see §\ref{sec:name}).
  \item[Illustration] This is a short/minimal example of the construction.
  \item[Berkeley ID] ~
  \item[Type]
  \item[Category] This is the resulting syntactic category of the construction.
  \item[FrameNet] The FrameNet frame invoked by the construction (if applicable).
  \item[Definition] A free text definition in Russian. The parts of the definition 
      corresponding to construction elements are tagged with the appropriate semantic role (see §\ref{sec:semroles})
  \item[Structure] A structure sketch
\end{description}

% what are "Keywords": parts of the construction

% what are "Common words": words found very frequently with this construction

% 


\end{document}
