\documentclass[a4paper,11pt, onecolumn,twoside]{article}

\usepackage{alltt}
\usepackage{polyglossia}
\setdefaultlanguage[variant=australian]{english}

\usepackage{fontspec}
\usepackage{xunicode}
\usepackage{longtable}

\setmainfont{Times New Roman}
\newfontfamily\ll[]{Linux Libertine O}
%\newfontfamily\udfont[Scale=1.3,Letters=SmallCaps]{Linux Libertine O}
\newfontfamily\udfont[Scale=1.3]{Linux Libertine O}
\newfontfamily\fsfont[Scale=MatchLowercase]{FreeSerif}

\usepackage[normalem]{ulem}
\usepackage{multirow}
\usepackage{multicol}
\usepackage[small,bf]{caption} % CHECK IF THIS IS OK
\usepackage[colorlinks=true,citecolor=black,linkcolor=black,urlcolor=blue]{hyperref}

\usepackage{subcaption}
\usepackage{booktabs}

\usepackage{tikz-dependency}
\usepackage{tikz-qtree}

\newcommand{\gmk}[1]{{\rm {\ll \textsc{#1}}}}
\newcommand{\kazakh}[1]{{\em #1}}
\newcommand{\gloss}[1]{`#1'}
\newcommand{\sgloss}[1]{\hspace{0.5em}`#1'}
%\newcommand{\sgloss}[1]{\hfill{}`#1'}
\newcommand{\tgloss}[1]{{\em #1}}
%\newcommand{\tag}[1]{\texttt{‹#1›}}
\newcommand{\tag}[1]{{\ll \textsc{‹#1›}}}
%\newcommand{\udtag}[1]{\texttt{@#1}}
\newcommand{\udtag}[1]{{\ll \textsc{#1}}}
\newcommand{\udlabel}[1]{{\udfont #1}}
\newcommand{\tilda}{{\fsfont ∼}}



\begin{document}

\section{Parts-of-speech}

The parts-of-speech in use are described in more detail here.\footnote{\url{http://universaldependencies.org/ru/pos/all.html}}

Open class words:

\begin{tabular}{ll}
   \gmk{adj} &    \\
   \multirow{2}{*}{\gmk{adv}} & «\emph{Германские войска \underline{быстро} продвигались вперёд} \\
                              & ~~\emph{и к 1942 году вышли к Северному Кавказу.}» \\
   \gmk{intj} &  \\
   \gmk{noun} &  \\
   \gmk{propn} &  \\
   \gmk{adv} &  \\
   \gmk{verb} &  \\
\end{tabular}

Closed class words:

\begin{tabular}{ll}
 \gmk{adp} &     \\
 \gmk{cconj} &    \\
 \gmk{det} &     \\
 \gmk{num} &     \\
 \gmk{part} &     \\
 \gmk{pron} &     \\
 \gmk{sconj} &     \\
\end{tabular}

Other:

\begin{tabular}{ll}
 \gmk{punct} & ~ \\
 \gmk{x} & ~ \\
\end{tabular}

\section{Morphological categories}

% Case 
% Tense
% Comparison

Morphological categories should be written in titlecase.

\scalebox{0.95}{
\begin{tabular}{lll}
\toprule
\multicolumn{3}{c}{\textbf{Number}} \\
\hline
Sg  & Singular & ~ \\
Pl  & Plural & ~ \\
\multicolumn{3}{c}{\textbf{Gender}} \\
\hline
Neut & Neuter & ~ \\
Masc & Masculine & ~ \\
Fem & Feminine& ~ \\
\multicolumn{3}{c}{\textbf{Person}} \\
\hline
1 & First person & ~ \\
2 & Second person & ~ \\
3 & Third person & ~ \\
\multicolumn{3}{c}{\textbf{Animacy}} \\
\hline
Anim & Animate & ~ \\
Inan & Inanimate & ~ \\
\multicolumn{3}{c}{\textbf{Case}} \\
\hline
Nom & Nominative &  \\
Gen & Genitive & ~ \\
Acc & Accusative &  \\
Dat & Dative & \emph{\underline{мне} нравится}, \emph{к \underline{родителям}} \\
Ins & Instrumental & ~ \\
Loc & Locative & ~ \\
\multicolumn{3}{c}{\textbf{Marginal cases}} \\
\hline
Voc & Vocative & \emph{ребят !} \\
Par & Partitive & \emph{чашка \underline{чаю}} \\
Adn & Adnumeral & \emph{два \underline{ряда́}} \\
Loc2 & 2nd Locative & \emph{в \underline{лесу}} \\
Acc2 & 2nd Accusative & \emph{пойти в \underline{депутаты}} \\
Dat2 & 2nd Dative & \emph{по \underline{скольку}, по \underline{стольку}} \\
\multicolumn{3}{c}{\textbf{Degree}} \\
\hline
Cmp & Comparative & ~ \\
\multicolumn{3}{c}{\textbf{Verbs}} \\
\hline
Inf & Infinitive & ~ \\
Pres & Present   & ~ \\
Past & Past & ~ \\
Fut & Future    & ~ \\
Bare & Bare form (хлоп, прыг, \ldots)    & ~ \\

Imper\footnote{This is equivalent to \texttt{Mood=Imp}} & Imperative & ~ \\
Imper2 & Second imperative & \emph{пойдёмте} \\
Imp & Imperfective & ~ \\
Perf & Perfective & ~ \\
Part &  Participle / Причастие & ~   \\
Pass & Passive & ~ \\
Act & Active & ~ \\
Short & Short form & \emph{принят}, \emph{холоден} \\
Conv & Verbal adverb / Деепричастие & ~ \\
\bottomrule
\end{tabular}
}

Consider using the following common abbreviations: 

\begin{tabular}{ll}
  Part.Past.Pass.Short & PartPast  \\
  Part.Pres.Act & PartPres \\
\end{tabular}

When there is more than one feature, they should be punctuated with `.', e.g. -Pl.Nom.

When referring to Person, the Number feature is also mandatory, e.g. -3.Pl

\section{Phrasal categories}

\begin{tabular}{lll}
\toprule
\multirow{2}{*}{\textbf{NP}} & Noun phrase & \\
                    & Мать за дочь сделала [NP домашнее задание] & \\
\midrule
\multirow{2}{*}{\textbf{VP}} & Verb phrase & \\
                    & Иван Иванович едва не [VP умер после такого розыгрыша] & \\
\midrule
\multirow{2}{*}{\textbf{AP}} & Adjectival phrase & \\
                    & & \\
\midrule
\multirow{2}{*}{\textbf{AdvP}} & Adverbial phrase & \\
                    & & \\
\midrule
\multirow{2}{*}{\textbf{PP}} & Prepositional phrase & \\
                    & А мы всё [PP без молока] и [PP без молока] & \\
\midrule
\multirow{2}{*}{\textbf{NumP}} & Numeral phrase & \\
                    & & \\
\midrule
\multirow{2}{*}{\textbf{XP}} & Any phrasal unit & \\
                    & Будь то [XP врач] или [XP учитель] & \\
\midrule
\multirow{2}{*}{\textbf{BareCl}} & Bare clause & \\
                    & Как будто [Cl появилась молния] & \\
\midrule
\multirow{2}{*}{\textbf{IndirCl}} & Indirect speech clause & \\
                    & Не знаю, [IndirCl пришёл \emph{ли} он] & \\
\midrule
\multirow{2}{*}{\textbf{Cl}} & Clause & \\
                    & [Cl Как будто появилась молния] & \\
\midrule
\multirow{2}{*}{\textbf{S}} & Sentence & \\
                    & & \\
\bottomrule
\end{tabular}

Phrase-level features should be indicated after a hyphen, `-'.

When to use a phrasal category?  --- When the constituent is not closed to modification.

\begin{figure}
\centering
\textbf{Open:}
\begin{tikzpicture}
\Tree [.далеко~до NP-Dat$\downarrow$ NP-Gen$\downarrow$ ]
\end{tikzpicture}
\begin{tikzpicture}
\Tree [.далеко~до [.NP-Dat [.N Улицам ] ] [.NP-Gen [.A идеальной ] [.N чистоты ] ] ]
\end{tikzpicture}

\textbf{Closed:}
\begin{tikzpicture}
\Tree [.N [.N N ] [.из из N-Pl.Gen ] ]
\end{tikzpicture}
\begin{tikzpicture}
\Tree [.N [.N король ] [.из из королей ] ]
\end{tikzpicture}

\end{figure}

\section{Discourse categories}

\begin{tabular}{lll}
\multirow{2}{*}{DiscC} & Discourse context & \\
                    & & \\
\multirow{2}{*}{DirSpeech} & Direct speech & \\
                    & & \\
\end{tabular}

\section{Dependency relations}

The dependency relations in the constructicon are based on Universal Dependencies.\footnote{\url{http://universaldependencies.org/ru/dep/all.html}}

\begin{figure}
\centering
        \begin{dependency}%[font = \small]
           \begin{deptext}[column sep=0.2cm]
                     ~ \&   \_ \& с \& \_                \\
                     POS:  \&  \gmk{PRON} \& \gmk{ADP} \& \_        \\
                     PronType:  \& \gmk{Pers} \& \_ \& \_ \\
                     Number: \& \gmk{Plur} \& \_ \& \_ \\
                     Case: \& \_ \& \_ \& \gmk{Ins} \\
                \end{deptext}
                \deproot[edge unit distance=1.3ex]{2}{\udlabel{x}}
                \depedge{4}{3}{\udlabel{case}}
                \depedge{2}{4}{\udlabel{nmod}}
        \end{dependency}
\end{figure}

\section{Semantic roles}
\label{sec:semroles}

Definitions of semantic roles have been taken from various sources.
% http://www-01.sil.org/linguistics/GlossaryOfLinguisticTerms/

\begin{longtable}{ p{.20\textwidth}  p{.80\textwidth} } 
\toprule
 \multirow{2}{*}{(Actant)} & Undetermined role \\ 
        & ~ \\
\midrule
 \multirow{2}{*}{Action} & Punctual  \\ 
        & ~ \\
\midrule
 \multirow{2}{*}{Activity} & Habitual or general \\ 
        & ~ \\
\midrule
 \multirow{2}{*}{Addressee} & Refers to the one who is addressed to in a speech situation. \\ 
        & ~ \\
\midrule
 \multirow{3}{*}{Agent} & Refers to the initiator of an event. \\ 
        & [Вася]$_{Agent}$ ловит рыбу. \\
        & Рыба ловится [Васей]$_{Agent}$ \\
\midrule
 \multirow{2}{*}{Associated} & Used in comitative construction \\ 
        & ~ \\
%\midrule
% \multirow{2}{*}{BaseOfPredication} & ~ \\ 
%        & ~ \\
\midrule
 \multirow{2}{*}{Beneficiary} & Refers to a referent which is advantaged or disadvantaged by an event. \\ 
        & Когда Саше было девять лет, мать купила [ему]$_{Beneficiary}$ кроссовки. \\
\midrule
 \multirow{2}{*}{Cause} & ~ \\ 
        & ~ \\
\midrule
 \multirow{2}{*}{Causee} & ~ \\ 
        & ~ \\
\midrule
 \multirow{2}{*}{Causer} & Refers to the referent which instigates an event rather than actually doing it. \\ 
        & ~ \\
\midrule
 \multirow{2}{*}{Circumstance} & ~ \\ 
        & ~ \\
\midrule
 \multirow{2}{*}{Class} & ~ \\ 
        & ~ \\
%\midrule
% \multirow{2}{*}{Concerning} & ~ \\ 
%        & ~ \\
\midrule
 \multirow{2}{*}{Condition} & ~ \\ 
        & ~ \\
%\midrule
% \multirow{2}{*}{Context} & ~ \\ 
%        & ~ \\
%\midrule
% \multirow{2}{*}{Corresponding} & ~ \\ 
%        & ~ \\
\midrule
 \multirow{2}{*}{Direction} & ~ \\ 
        & ~ \\
\midrule
 \multirow{2}{*}{Distance} & ~ \\ 
        & ~ \\
\midrule
 \multirow{2}{*}{Entity} & ~ \\ 
        & ~ \\
\midrule
 \multirow{2}{*}{Evaluation} & ~ \\  
        & В этом поступке не было [ничего необычного]$_{Evaluation}$ \\
\midrule
 \multirow{2}{*}{Event} & ~ \\ 
        & ~ \\
\midrule
 \multirow{2}{*}{Experiencer} & Refers to an entity that receives a sensory impression, or in some other way is the locus of some event or activity that involves neither volition nor a change of state. \\ 
        & ~ \\
\midrule
 \multirow{2}{*}{Function} & ~ \\ 
        & Когда я говорю «право» я имею в виду [«лево»]$_{Function}$. \\
\midrule
 \multirow{2}{*}{Goal} & Refers to the place to which something moves, or the thing towards which an action is directed. \\ 
%        & ~ \\
\midrule
 \multirow{2}{*}{Goer} & ~ \\ 
        & ~ \\
\midrule
 \multirow{2}{*}{Instrument} & Refers to an inanimate thing that an agent uses to implement an event. It is the stimulus or immediate physical cause of an event. \\ 
        & Он написал пиьсмо [карандашом]$_{Instrument}$. \\
\midrule
 \multirow{2}{*}{Item} & Refers to \ldots CHECK THIS \\ 
        & ~ \\
\midrule
 \multirow{2}{*}{Limit} &  ~ \\
        & ~ \\
\midrule
 \multirow{2}{*}{Location} & Refers to the location or spatial orientation of a state or action. It does not imply any motion. \\ 
        & Иван нашел лягушку [в лесу]$_{Location}$. \\
%\midrule
% \multirow{2}{*}{Locative} &  ~ \\
%        & ~ \\
\midrule
 \multirow{2}{*}{Manner} & Notes how the action, experience, or process of an event is carried out. \\ 
        & ~ \\
\midrule
 \multirow{2}{*}{Material} & ~ \\ 
        & ~ \\
\midrule
 \multirow{2}{*}{Measure} & Notes the quantification of an event. \\ 
        & ~ \\
\midrule
 \multirow{2}{*}{Motivation} &  \\ 
        & ~ \\
%\midrule
% \multirow{2}{*}{Object} & ~ \\ 
%        & ~ \\
\midrule
 \multirow{2}{*}{Participant} & ~ \\ 
        & ~ \\
\midrule
 \multirow{2}{*}{Path} & ~ \\ 
        & ~ \\
\midrule
 \multirow{2}{*}{Parameter} & \\
        & ~ \\
\midrule
 \multirow{2}{*}{Patient} & Refers to the surface object of the verb in a sentence. \\ 
        & ~ \\
\midrule
 \multirow{2}{*}{Phenomenon} & ~ \\  % Probably not used 
        & ~ \\
\midrule
 \multirow{2}{*}{Property} & ~ \\ 
        & ~ \\
\midrule
 \multirow{2}{*}{Purpose} & ~ \\ 
        & ~ \\
\midrule
 \multirow{2}{*}{Protagonist} & ~ \\ % General agent-like participant
        & ~ \\
\midrule
 \multirow{2}{*}{Quantity} & ~ \\  % 'Quantitative' in Swedish constructicon
        & ~ \\
\midrule
 \multirow{2}{*}{Recipient} & Refers to a referent that is conscious of being affected by the state or action identified by the verb. \\ 
        & ~ \\
%\midrule
% \multirow{2}{*}{Requirement} & ~ \\ 
%        & ~ \\
\midrule
 \multirow{2}{*}{Result} & ~ \\ 
        & ~ \\
\midrule
 \multirow{2}{*}{Set} & ~ \\ 
        & ~ \\
\midrule
 \multirow{2}{*}{Situation} & ~ \\ 
        & ~ \\
\midrule
 \multirow{2}{*}{Source} & Refers to the (a) the place of origin, (b) the entity from which a physical sensation comes, (c) the original owner in a transfer. \\ 
        & ~ \\
\midrule
 \multirow{2}{*}{Speaker} & ~ \\ 
        & ~ \\
\midrule
 \multirow{2}{*}{Standard} & Base of comparison. Also used for identificational sentences like \emph{Он же мальчик!} CHECK \\ % Was 'Corresponding'
        & ~ \\
\midrule
 \multirow{2}{*}{State} & ~ \\ 
        & ~ \\
\midrule
 \multirow{2}{*}{Theme} & ~ \\ 
        & ~ \\
\midrule
 \multirow{2}{*}{Topic} & ~ \\  % Was 'Concerning', pragmatic element?
        & ~ \\
%\midrule
% \multirow{2}{*}{Trajector} & ~ \\ 
%        & ~ \\
\midrule
 \multirow{2}{*}{Undergoer} & ~ \\ % Very general patient-like participant
        & ~ \\
\bottomrule
\caption{List of semantic roles}
\label{table:semroles}
\end{longtable}

\section{Naming constructions}
\label{sec:name}

% when X/Y/NP/etc.

If the construction forms a complete utterance, such as `как бы не VP-Inf!' then the 
name of the construction should be in sentence case, e.g. `Как бы не VP-Inf!'

If the construction is an for exclamation or question, then it should include the 
relevant punctuation mark: !, ?

When there is a construction that has two possibilities, then the name should 
include the most frequent. For example: instead of `NP-Nom говорить|выступать перед NP-Dat',
the construction should be named `NP-Nom говорить перед NP-Dat' and `выступать' should
be added to the \texttt{Common words} field, possibly with a note in the description.

If you need to number a constituent, the numeral should come straight after the phrasal
category in superscript, e.g. `То ли NP¹, то ли NP²'. This allows us to distinguish indexing
from morphological features.

\section{The structure of the constructicon}

\begin{description}
  \item[Name] This is the name of the construction (see §\ref{sec:name}).
  \item[Illustration] This is a short/minimal example of the construction.
  \item[Berkeley ID] ~
  \item[Type]
  \item[Category] This is the resulting syntactic category of the construction.
  \item[FrameNet] The FrameNet frame invoked by the construction (if applicable).
  \item[Definition] A free text definition in Russian. The parts of the definition 
      corresponding to construction elements are tagged with the appropriate semantic role (see §\ref{sec:semroles})
  \item[Structure] A structure sketch
\end{description}

% what are "Keywords": parts of the construction

% what are "Common words": words found very frequently with this construction

% 


\end{document}
